% Options for packages loaded elsewhere
\PassOptionsToPackage{unicode}{hyperref}
\PassOptionsToPackage{hyphens}{url}
\PassOptionsToPackage{dvipsnames,svgnames,x11names}{xcolor}
%
\documentclass[
]{article}

\usepackage{amsmath,amssymb}
\usepackage{iftex}
\ifPDFTeX
  \usepackage[T1]{fontenc}
  \usepackage[utf8]{inputenc}
  \usepackage{textcomp} % provide euro and other symbols
\else % if luatex or xetex
  \usepackage{unicode-math}
  \defaultfontfeatures{Scale=MatchLowercase}
  \defaultfontfeatures[\rmfamily]{Ligatures=TeX,Scale=1}
\fi
\usepackage{lmodern}
\ifPDFTeX\else  
    % xetex/luatex font selection
\fi
% Use upquote if available, for straight quotes in verbatim environments
\IfFileExists{upquote.sty}{\usepackage{upquote}}{}
\IfFileExists{microtype.sty}{% use microtype if available
  \usepackage[]{microtype}
  \UseMicrotypeSet[protrusion]{basicmath} % disable protrusion for tt fonts
}{}
\makeatletter
\@ifundefined{KOMAClassName}{% if non-KOMA class
  \IfFileExists{parskip.sty}{%
    \usepackage{parskip}
  }{% else
    \setlength{\parindent}{0pt}
    \setlength{\parskip}{6pt plus 2pt minus 1pt}}
}{% if KOMA class
  \KOMAoptions{parskip=half}}
\makeatother
\usepackage{xcolor}
\setlength{\emergencystretch}{3em} % prevent overfull lines
\setcounter{secnumdepth}{-\maxdimen} % remove section numbering
% Make \paragraph and \subparagraph free-standing
\makeatletter
\ifx\paragraph\undefined\else
  \let\oldparagraph\paragraph
  \renewcommand{\paragraph}{
    \@ifstar
      \xxxParagraphStar
      \xxxParagraphNoStar
  }
  \newcommand{\xxxParagraphStar}[1]{\oldparagraph*{#1}\mbox{}}
  \newcommand{\xxxParagraphNoStar}[1]{\oldparagraph{#1}\mbox{}}
\fi
\ifx\subparagraph\undefined\else
  \let\oldsubparagraph\subparagraph
  \renewcommand{\subparagraph}{
    \@ifstar
      \xxxSubParagraphStar
      \xxxSubParagraphNoStar
  }
  \newcommand{\xxxSubParagraphStar}[1]{\oldsubparagraph*{#1}\mbox{}}
  \newcommand{\xxxSubParagraphNoStar}[1]{\oldsubparagraph{#1}\mbox{}}
\fi
\makeatother

\usepackage{color}
\usepackage{fancyvrb}
\newcommand{\VerbBar}{|}
\newcommand{\VERB}{\Verb[commandchars=\\\{\}]}
\DefineVerbatimEnvironment{Highlighting}{Verbatim}{commandchars=\\\{\}}
% Add ',fontsize=\small' for more characters per line
\usepackage{framed}
\definecolor{shadecolor}{RGB}{241,243,245}
\newenvironment{Shaded}{\begin{snugshade}}{\end{snugshade}}
\newcommand{\AlertTok}[1]{\textcolor[rgb]{0.68,0.00,0.00}{#1}}
\newcommand{\AnnotationTok}[1]{\textcolor[rgb]{0.37,0.37,0.37}{#1}}
\newcommand{\AttributeTok}[1]{\textcolor[rgb]{0.40,0.45,0.13}{#1}}
\newcommand{\BaseNTok}[1]{\textcolor[rgb]{0.68,0.00,0.00}{#1}}
\newcommand{\BuiltInTok}[1]{\textcolor[rgb]{0.00,0.23,0.31}{#1}}
\newcommand{\CharTok}[1]{\textcolor[rgb]{0.13,0.47,0.30}{#1}}
\newcommand{\CommentTok}[1]{\textcolor[rgb]{0.37,0.37,0.37}{#1}}
\newcommand{\CommentVarTok}[1]{\textcolor[rgb]{0.37,0.37,0.37}{\textit{#1}}}
\newcommand{\ConstantTok}[1]{\textcolor[rgb]{0.56,0.35,0.01}{#1}}
\newcommand{\ControlFlowTok}[1]{\textcolor[rgb]{0.00,0.23,0.31}{\textbf{#1}}}
\newcommand{\DataTypeTok}[1]{\textcolor[rgb]{0.68,0.00,0.00}{#1}}
\newcommand{\DecValTok}[1]{\textcolor[rgb]{0.68,0.00,0.00}{#1}}
\newcommand{\DocumentationTok}[1]{\textcolor[rgb]{0.37,0.37,0.37}{\textit{#1}}}
\newcommand{\ErrorTok}[1]{\textcolor[rgb]{0.68,0.00,0.00}{#1}}
\newcommand{\ExtensionTok}[1]{\textcolor[rgb]{0.00,0.23,0.31}{#1}}
\newcommand{\FloatTok}[1]{\textcolor[rgb]{0.68,0.00,0.00}{#1}}
\newcommand{\FunctionTok}[1]{\textcolor[rgb]{0.28,0.35,0.67}{#1}}
\newcommand{\ImportTok}[1]{\textcolor[rgb]{0.00,0.46,0.62}{#1}}
\newcommand{\InformationTok}[1]{\textcolor[rgb]{0.37,0.37,0.37}{#1}}
\newcommand{\KeywordTok}[1]{\textcolor[rgb]{0.00,0.23,0.31}{\textbf{#1}}}
\newcommand{\NormalTok}[1]{\textcolor[rgb]{0.00,0.23,0.31}{#1}}
\newcommand{\OperatorTok}[1]{\textcolor[rgb]{0.37,0.37,0.37}{#1}}
\newcommand{\OtherTok}[1]{\textcolor[rgb]{0.00,0.23,0.31}{#1}}
\newcommand{\PreprocessorTok}[1]{\textcolor[rgb]{0.68,0.00,0.00}{#1}}
\newcommand{\RegionMarkerTok}[1]{\textcolor[rgb]{0.00,0.23,0.31}{#1}}
\newcommand{\SpecialCharTok}[1]{\textcolor[rgb]{0.37,0.37,0.37}{#1}}
\newcommand{\SpecialStringTok}[1]{\textcolor[rgb]{0.13,0.47,0.30}{#1}}
\newcommand{\StringTok}[1]{\textcolor[rgb]{0.13,0.47,0.30}{#1}}
\newcommand{\VariableTok}[1]{\textcolor[rgb]{0.07,0.07,0.07}{#1}}
\newcommand{\VerbatimStringTok}[1]{\textcolor[rgb]{0.13,0.47,0.30}{#1}}
\newcommand{\WarningTok}[1]{\textcolor[rgb]{0.37,0.37,0.37}{\textit{#1}}}

\providecommand{\tightlist}{%
  \setlength{\itemsep}{0pt}\setlength{\parskip}{0pt}}\usepackage{longtable,booktabs,array}
\usepackage{calc} % for calculating minipage widths
% Correct order of tables after \paragraph or \subparagraph
\usepackage{etoolbox}
\makeatletter
\patchcmd\longtable{\par}{\if@noskipsec\mbox{}\fi\par}{}{}
\makeatother
% Allow footnotes in longtable head/foot
\IfFileExists{footnotehyper.sty}{\usepackage{footnotehyper}}{\usepackage{footnote}}
\makesavenoteenv{longtable}
\usepackage{graphicx}
\makeatletter
\newsavebox\pandoc@box
\newcommand*\pandocbounded[1]{% scales image to fit in text height/width
  \sbox\pandoc@box{#1}%
  \Gscale@div\@tempa{\textheight}{\dimexpr\ht\pandoc@box+\dp\pandoc@box\relax}%
  \Gscale@div\@tempb{\linewidth}{\wd\pandoc@box}%
  \ifdim\@tempb\p@<\@tempa\p@\let\@tempa\@tempb\fi% select the smaller of both
  \ifdim\@tempa\p@<\p@\scalebox{\@tempa}{\usebox\pandoc@box}%
  \else\usebox{\pandoc@box}%
  \fi%
}
% Set default figure placement to htbp
\def\fps@figure{htbp}
\makeatother

\makeatletter
\@ifpackageloaded{caption}{}{\usepackage{caption}}
\AtBeginDocument{%
\ifdefined\contentsname
  \renewcommand*\contentsname{Table of contents}
\else
  \newcommand\contentsname{Table of contents}
\fi
\ifdefined\listfigurename
  \renewcommand*\listfigurename{List of Figures}
\else
  \newcommand\listfigurename{List of Figures}
\fi
\ifdefined\listtablename
  \renewcommand*\listtablename{List of Tables}
\else
  \newcommand\listtablename{List of Tables}
\fi
\ifdefined\figurename
  \renewcommand*\figurename{Figure}
\else
  \newcommand\figurename{Figure}
\fi
\ifdefined\tablename
  \renewcommand*\tablename{Table}
\else
  \newcommand\tablename{Table}
\fi
}
\@ifpackageloaded{float}{}{\usepackage{float}}
\floatstyle{ruled}
\@ifundefined{c@chapter}{\newfloat{codelisting}{h}{lop}}{\newfloat{codelisting}{h}{lop}[chapter]}
\floatname{codelisting}{Listing}
\newcommand*\listoflistings{\listof{codelisting}{List of Listings}}
\makeatother
\makeatletter
\makeatother
\makeatletter
\@ifpackageloaded{caption}{}{\usepackage{caption}}
\@ifpackageloaded{subcaption}{}{\usepackage{subcaption}}
\makeatother

\usepackage{bookmark}

\IfFileExists{xurl.sty}{\usepackage{xurl}}{} % add URL line breaks if available
\urlstyle{same} % disable monospaced font for URLs
\hypersetup{
  pdftitle={KD-GAT Pipeline Report},
  pdfauthor={Auto-generated},
  colorlinks=true,
  linkcolor={blue},
  filecolor={Maroon},
  citecolor={Blue},
  urlcolor={Blue},
  pdfcreator={LaTeX via pandoc}}


\title{KD-GAT Pipeline Report}
\author{Auto-generated}
\date{2026-02-25}

\begin{document}
\maketitle


\subsection{Overview}\label{overview}

Automated report from the KD-GAT intrusion detection pipeline. Queries
the project database and Snakemake benchmark TSVs.

\phantomsection\label{setup}
\begin{Shaded}
\begin{Highlighting}[]
\ImportTok{import}\NormalTok{ json}
\ImportTok{import}\NormalTok{ sqlite3}
\ImportTok{from}\NormalTok{ pathlib }\ImportTok{import}\NormalTok{ Path}

\ImportTok{import}\NormalTok{ pandas }\ImportTok{as}\NormalTok{ pd}

\NormalTok{DB\_PATH }\OperatorTok{=}\NormalTok{ Path(}\StringTok{"../data/project.db"}\NormalTok{)}
\NormalTok{EXP }\OperatorTok{=}\NormalTok{ Path(}\StringTok{"../experimentruns"}\NormalTok{)}

\NormalTok{conn }\OperatorTok{=}\NormalTok{ sqlite3.}\ExtensionTok{connect}\NormalTok{(DB\_PATH)}
\NormalTok{conn.row\_factory }\OperatorTok{=}\NormalTok{ sqlite3.Row}
\end{Highlighting}
\end{Shaded}

\subsection{Dataset Summary}\label{dataset-summary}

\phantomsection\label{datasets}
\begin{Shaded}
\begin{Highlighting}[]
\NormalTok{datasets }\OperatorTok{=}\NormalTok{ pd.read\_sql(}\StringTok{"SELECT name, domain, num\_samples, num\_unique\_ids FROM datasets"}\NormalTok{, conn)}
\NormalTok{datasets}
\end{Highlighting}
\end{Shaded}

\subsection{Leaderboard (F1)}\label{leaderboard-f1}

\phantomsection\label{leaderboard}
\begin{Shaded}
\begin{Highlighting}[]
\NormalTok{leaderboard }\OperatorTok{=}\NormalTok{ pd.read\_sql(}\StringTok{"""}
\StringTok{    SELECT r.dataset, r.stage, m.model, m.scenario,}
\StringTok{           ROUND(m.value, 4) AS f1}
\StringTok{    FROM metrics m}
\StringTok{    JOIN runs r ON r.run\_id = m.run\_id}
\StringTok{    WHERE m.metric\_name = \textquotesingle{}f1\textquotesingle{}}
\StringTok{    ORDER BY m.value DESC}
\StringTok{    LIMIT 20}
\StringTok{"""}\NormalTok{, conn)}
\NormalTok{leaderboard}
\end{Highlighting}
\end{Shaded}

\subsection{Benchmark Summary}\label{benchmark-summary}

Wall time, peak RSS, and CPU time from Snakemake \texttt{benchmark:}
TSVs.

\phantomsection\label{benchmarks}
\begin{Shaded}
\begin{Highlighting}[]
\NormalTok{benchmarks }\OperatorTok{=}\NormalTok{ []}
\ControlFlowTok{for}\NormalTok{ tsv }\KeywordTok{in}\NormalTok{ EXP.rglob(}\StringTok{"benchmark.tsv"}\NormalTok{):}
    \ControlFlowTok{try}\NormalTok{:}
\NormalTok{        df }\OperatorTok{=}\NormalTok{ pd.read\_csv(tsv, sep}\OperatorTok{=}\StringTok{"}\CharTok{\textbackslash{}t}\StringTok{"}\NormalTok{)}
\NormalTok{        parts }\OperatorTok{=}\NormalTok{ tsv.relative\_to(EXP).parts}
\NormalTok{        df[}\StringTok{"dataset"}\NormalTok{] }\OperatorTok{=}\NormalTok{ parts[}\DecValTok{0}\NormalTok{]}
\NormalTok{        df[}\StringTok{"run"}\NormalTok{] }\OperatorTok{=}\NormalTok{ parts[}\DecValTok{1}\NormalTok{]}
\NormalTok{        benchmarks.append(df)}
    \ControlFlowTok{except} \PreprocessorTok{Exception}\NormalTok{:}
        \ControlFlowTok{pass}

\ControlFlowTok{if}\NormalTok{ benchmarks:}
\NormalTok{    bench\_df }\OperatorTok{=}\NormalTok{ pd.concat(benchmarks, ignore\_index}\OperatorTok{=}\VariableTok{True}\NormalTok{)}
\NormalTok{    bench\_df[[}\StringTok{"dataset"}\NormalTok{, }\StringTok{"run"}\NormalTok{, }\StringTok{"s"}\NormalTok{, }\StringTok{"max\_rss"}\NormalTok{, }\StringTok{"cpu\_time"}\NormalTok{]].}\BuiltInTok{round}\NormalTok{(}\DecValTok{1}\NormalTok{)}
\ControlFlowTok{else}\NormalTok{:}
    \BuiltInTok{print}\NormalTok{(}\StringTok{"No benchmark TSVs found yet. Run the pipeline first."}\NormalTok{)}
\end{Highlighting}
\end{Shaded}

\subsection{Teacher vs Student
Comparison}\label{teacher-vs-student-comparison}

\phantomsection\label{comparison}
\begin{Shaded}
\begin{Highlighting}[]
\NormalTok{comparison }\OperatorTok{=}\NormalTok{ pd.read\_sql(}\StringTok{"""}
\StringTok{    SELECT r.dataset,}
\StringTok{           r.model\_size,}
\StringTok{           r.use\_kd,}
\StringTok{           m.metric\_name,}
\StringTok{           ROUND(m.value, 4) AS value}
\StringTok{    FROM metrics m}
\StringTok{    JOIN runs r ON r.run\_id = m.run\_id}
\StringTok{    WHERE m.metric\_name IN (\textquotesingle{}f1\textquotesingle{}, \textquotesingle{}accuracy\textquotesingle{}, \textquotesingle{}auc\textquotesingle{}, \textquotesingle{}mcc\textquotesingle{})}
\StringTok{      AND r.stage = \textquotesingle{}evaluation\textquotesingle{}}
\StringTok{    ORDER BY r.dataset, m.metric\_name, r.model\_size}
\StringTok{"""}\NormalTok{, conn)}
\ControlFlowTok{if} \KeywordTok{not}\NormalTok{ comparison.empty:}
\NormalTok{    comparison.pivot\_table(}
\NormalTok{        index}\OperatorTok{=}\NormalTok{[}\StringTok{"dataset"}\NormalTok{, }\StringTok{"metric\_name"}\NormalTok{],}
\NormalTok{        columns}\OperatorTok{=}\NormalTok{[}\StringTok{"model\_size"}\NormalTok{, }\StringTok{"use\_kd"}\NormalTok{],}
\NormalTok{        values}\OperatorTok{=}\StringTok{"value"}\NormalTok{,}
\NormalTok{    )}
\ControlFlowTok{else}\NormalTok{:}
    \BuiltInTok{print}\NormalTok{(}\StringTok{"No evaluation runs found yet."}\NormalTok{)}
\end{Highlighting}
\end{Shaded}

\phantomsection\label{cleanup}
\begin{Shaded}
\begin{Highlighting}[]
\NormalTok{conn.close()}
\end{Highlighting}
\end{Shaded}





\end{document}
